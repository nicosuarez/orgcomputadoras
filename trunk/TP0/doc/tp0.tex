%%%%%%%%%%%%%%%%%%%%%%%%%%%%%%%%%%%%%%%%%%%%%%%%%%%%%%%%%%%%%%%%%%%%%%%%%%%%%%%
% Definici�n del tipo de documento.                                           %
% Posibles tipos de papel: a4paper, letterpaper, legalpapper                  %
% Posibles tama�os de letra: 10pt, 11pt, 12pt                                 %
% Posibles clases de documentos: article, report, book, slides                %
%%%%%%%%%%%%%%%%%%%%%%%%%%%%%%%%%%%%%%%%%%%%%%%%%%%%%%%%%%%%%%%%%%%%%%%%%%%%%%%
\documentclass[a4paper,10pt]{article}


%%%%%%%%%%%%%%%%%%%%%%%%%%%%%%%%%%%%%%%%%%%%%%%%%%%%%%%%%%%%%%%%%%%%%%%%%%%%%%%
% Los paquetes permiten ampliar las capacidades de LaTeX.                     %
%%%%%%%%%%%%%%%%%%%%%%%%%%%%%%%%%%%%%%%%%%%%%%%%%%%%%%%%%%%%%%%%%%%%%%%%%%%%%%%

% Paquete para inclusi�n de gr�ficos.
\usepackage{graphicx}

% Paquete para definir la codificaci�n del conjunto de caracteres usado
% (latin1 es ISO 8859-1).
\usepackage[latin1]{inputenc}

% Paquete para definir el idioma usado.
\usepackage[spanish]{babel}


% T�tulo principal del documento.
\title{		\textbf{Trabajo pr�ctico 0: Infraestructura b�sica}}

% Informaci�n sobre los autores.
\author{		Roberto Herman, \textit{Padr�n Nro. 84.803}                     \\
            \texttt{ berta1108@gmail.com }                                              \\
            Mat�as Waisgold, \textit{Padr�n Nro. 00.000}                     \\
            \texttt{ mwaisgold@gmail.com }                                              \\
            Nicol�s Suarez, \textit{Padr�n Nro. 00.000}                     \\
            \texttt{ direcci�n de e-mail }                                              \\
            Federico , \textit{Padr�n Nro. 00.000}                     \\
            \texttt{ direcci�n de e-mail }                                              \\[2.5ex]
            \normalsize{Grupo Nro. 0 - 2do. Cuatrimestre de 2009}                       \\
            \normalsize{66.20 Organizaci�n de Computadoras}                             \\
            \normalsize{Facultad de Ingenier�a, Universidad de Buenos Aires}            \\
       }
\date{}



\begin{document}

% Inserta el t�tulo.
\maketitle

% Quita el n�mero en la primer p�gina.
\thispagestyle{empty}

% Resumen
\begin{abstract}
El presente trabajo pr�ctico consiste en una introducci�n a la utilizaci�n de la infraestructura utilizada para el desarrollo de los trabajos pr�cticos de esta materia. La misma ser� llevada a cabo a trav�s de el desarrollo de una una versi�n reducida del comando \textbf{tail} de UNIX. El comando ser� codificado en lenguaje C, y se compilar� para funcionar en el emulador de \textbf{MIPS32 NetBsd}.
\end{abstract}


\section{Introducci�n}

Intro al tp.


\section{Dise�o e implementaci�n}

\subsection{M�dulo principal}

El m�dulo principal main.c contiene la validaci�n de par�metros de entrada
y el parseo de los mismos...

\subsection{Parser de l�neas}

Bla.

\section{Comandos}

En esta secci�n se describen los comandos necesarios para la compilaci�n de
la aplicaci�n.

\subsection{Generando el programa ejecutable}

gcc -Wall -o tp0 main.c parser.c

\subsection{Obteniendo el c�digo MIPS generado por el compilador}

gcc -Wall -O0 -S -mrnames main.c parser.c

\section{C�digo fuente}

\subsection{Lenguaje C}

\begin{verbatim}
Ac� codigo C.
\end{verbatim}

\subsection{Lenguaje MIPS}

\begin{verbatim}
Ac� codigo MIPS.
\end{verbatim}

\section{Corridas de prueba}

Definir qu� pruebas vamos a hacer, el enunciado no especifica cuales.

\section{Conclusiones}

Alto chamullo.


% Citas bibliogr�ficas.
\begin{thebibliography}{99}

Aca ir�a Bibliografia.

\end{thebibliography}

\end{document}
